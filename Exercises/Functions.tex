\documentclass[11pt]{book}
\usepackage{amsfonts}
\usepackage{amsmath}
\usepackage{amssymb}
\usepackage{amsthm}

\usepackage{amsthm}
\newtheorem{Def}{Definition}
\newtheorem{Lemma}{Lemma}
\newtheorem{Thm}{Theorem}
\newtheorem{Coro}{Corollary}
\newtheorem{Prop}{Proposition}


\newcommand{\bb}[1]{\mathbb{#1}}
\newcommand{\ra}{\rightarrow}
\newcommand{\Ra}{\Rightarrow}

\title{Functions}
\author{NzSN}
\date{2024 February 29}

\setlength{\parindent}{0pt}
\setlength{\topmargin}{-.3 in}
\setlength{\oddsidemargin}{0in}
\setlength{\evensidemargin}{0in}
\setlength{\textheight}{9.in}
\setlength{\textwidth}{6.5in}
\pagestyle{empty}

\begin{document}
\maketitle

\vspace{0.2 cm}
\section{Solutions}

\begin{enumerate}
  \item[1]
        Prove Proposition 3.6.
        \begin{proof}
          By proposition 3.5, if there exists function $G$ such that $G \circ (g \circ f) = id_{\bf{X}}$ and
          $(g \circ f) \circ G = id_{\bf{V}}$ then $(g \circ f)$ is bijective. And by definition of Inverse
          Functions we know $G$ is the reverse function of $(g \circ f)$.

          Due to $f$ and $g$ are bijective so exist $f^{-1}$ and $g^{-1}$ and due to
          $\circ$ is associative so we have
          $$f^{-1} \circ g^{-1} (g \circ f) = f^{-1} \circ (g^{-1} \circ g) \circ f =
          f^{-1} \circ id_{\bf{Y}} \circ f = f^{-1} \circ f = id_{\bf{X}}$$
          similarly
          $$(g \circ f) \circ f^{-1} \circ g^{-1} = id_{\bf{V}}$$.

          Hence, $(g \circ f)$ is bijective and $(g \circ f)^{-1} = f^{-1} \circ g^{-1}$
        \end{proof}

  \item[3]
        \begin{enumerate}
          \item[(a)]
                \begin{proof}
                Because of $f: \bf{X} \ra \bf{Y}$ and $g: \bf{Y} \ra \bf{V}$ so
                $$\exists \tilde{f}: im(f) \ra \bb{X}: \tilde{f} \circ f = id_{\bf{X}}$$
                $$\exists \tilde{f}: im(f) \ra \bb{X}: \tilde{f} \circ f = id_{\bf{X}}$$

                which implies that

                $$(\tilde{f} \circ \tilde{g}) \circ (g \circ f) = id_{\bf{X}}$$

                so $(f \circ g)$ is injective. Prove of surjective is similar.
                \end{proof}
          \item[(b)]
                \begin{proof}
                If $f$ is injective then there exists a unique function $\tilde{f}: im(f) \ra X$
                such that $\tilde{f} \circ f = id_{\bf{X}}$. Then define function $\Tilde{f}$ in a
                way that $\Tilde{f}\vert_{im(f)} = \tilde{f}$. Finally, we got a
                function that $\Tilde{f} \circ f = id_{\bf{x}}$
                \end{proof}
          \item[(c)]
                \begin{proof}
                Similar to (b).
                \end{proof}
        \end{enumerate}
  \item[6]
        \begin{proof}
          Let $f$ be a function that
          $$f: \cal{P}(\bf{X}) \ra \{0,1\}^{\bf{X}}, A \mapsto \cal{X}_{A}$$
          $\mathcal{X}$ is a function on $\bf{X}$ taking the values is $\{0,1\}$.
          Exists $\bf{A} \subseteq \bf{X}$ that $\cal{X}_{\bf{A}}$ send all element
          of $\bf{A}$ to 1. Due to domain of $f$ is all subset of $\bb{X}$ we must have
          $\bf{A} \in \bf{X}$ so $f$ is surjective.\\

          Define a function
          $$\Tilde{f}: \{0,1\}^{\bf{X}} \ra \cal{P}(\bf{X}),\cal{X}_{\bf{A}} \mapsto A$$
          then $\Tilde{f} \circ f = id_{\cal{P}(\bf{X})}$ so $f$ is injective. Hence $f$ is
          bijective.
        \end{proof}
  \item[7]
        \begin{enumerate}
          \item[(a)]
                Both domain and codomain of those two functions are same. Then if we prove that their graph are same then they are equal.

                Choose arbitary element $x \in \bf{A}$ we have $f\vert_{A}(x) = (f \circ i)(x)$ directly from definition of restriction and inclusion.
          \item[(b)]

        \end{enumerate}
\end{enumerate}
\end{document}
