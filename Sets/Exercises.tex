\documentclass[11pt]{book}
\usepackage{amsfonts}
\usepackage{amsmath}
\usepackage{amssymb}

\usepackage{amsthm}
\newtheorem{Def}{Definition}
\newtheorem{Lemma}{Lemma}
\newtheorem{Thm}{Theorem}
\newtheorem{Coro}{Corollary}
\newtheorem{Prop}{Proposition}


\newcommand{\bb}[1]{\mathbb{#1}}
\newcommand{\ra}{\rightarrow}
\newcommand{\Ra}{\Rightarrow}

\title{Sets}
\author{NzSN}
\date{February 26}

\setlength{\parindent}{0pt}
\setlength{\topmargin}{-.3 in}
\setlength{\oddsidemargin}{0in}
\setlength{\evensidemargin}{0in}
\setlength{\textheight}{9.in}
\setlength{\textwidth}{6.5in}
\pagestyle{empty}

\begin{document}
\maketitle

\vspace{0.2 cm}
\section{Solutions}

\begin{enumerate}
\item[1]
        Let $\bb{X}$,$\bb{Y}$ and $\bb{Z}$ bet sets.
        Prove the transitivity of inclusion, that is,
        $$(\bb{X} \subseteq \bb{Y}) \land (\bb{Y} \subseteq \bb{Z}) \Ra \bb{X} \subseteq \bb{Z}$$
\begin{proof}
  Suppose exists an arbitary element $x \in \bb{X}$ then $x \in \bb{Y}$ by the definition of $\subseteq$,
  Hence, $x \in \bb{Z}$ cause $\bb{Y} \subseteq \bb{Z}$.
\end{proof}

\item[2]
        Verify the claims of Proposition 2.4
\begin{proof}
Suppose that $\bb{X}$, $\bb{Y}$, $\bb{Z}$ are subset of $\bb{U}$.
\begin{enumerate}
  \item[(i)]
        $\bb{X} \cup \bb{Y} := \{ x \in \bb{U} : (x \in \bb{X}) \lor (x \in \bb{Y}) \}$ and
        $\bb{Y} \cup \bb{Y} := \{ x \in \bb{U} : (x \in \bb{Y}) \lor (x \in \bb{X}) \}$.
        Clearly, those two expression are equal by their definition. It's similar to $\bb{X} \cap \bb{Y}$.
  \item[(ii)]
        $\cap$ and $\cup$ are associativity due to associativity of $\land$ and $\lor$
  \item[(iii)]
        First to prove
        $$\bb{X} \cup (\bb{Y} \cap \bb{Z}) \Ra
        (\bb{X} \cup \bb{Y}) \cap (\bb{X} \cup \bb{Z})$$

        By definition,

        $$\bb{X} \cup (\bb{Y} \cap \bb{Z}) \Ra
        \forall x \in \bb{U}: (x \in \bb{X}) \lor (x \in \bb{Y} \land x \in \bb{Z})
        \Ra \forall x \in \bb{U}: (x \in \bb{X} \lor x \in \bb{Y}) \land (x \in \bb{X} \lor x \in \bb{Z})$$

        prove of another direction is similarly.
  \item[(iv)]
        directly from definitions.
\end{enumerate}

\end{proof}

\end{enumerate}


\end{document}
