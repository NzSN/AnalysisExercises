\documentclass[11pt]{book}
\usepackage{amsfonts}
\usepackage{amsmath}
\usepackage{amssymb}
\usepackage{amsthm}

\usepackage{amsthm}
\newtheorem{Def}{Definition}
\newtheorem{Lemma}{Lemma}
\newtheorem{Thm}{Theorem}
\newtheorem{Coro}{Corollary}
\newtheorem{Prop}{Proposition}


\newcommand{\bb}[1]{\mathbb{#1}}
\newcommand{\ra}{\rightarrow}
\newcommand{\Ra}{\Rightarrow}

\title{Sets}
\author{NzSN}
\date{February 26}

\setlength{\parindent}{0pt}
\setlength{\topmargin}{-.3 in}
\setlength{\oddsidemargin}{0in}
\setlength{\evensidemargin}{0in}
\setlength{\textheight}{9.in}
\setlength{\textwidth}{6.5in}
\pagestyle{empty}

\begin{document}
\maketitle

\vspace{0.2 cm}
\section{Solutions}

\begin{enumerate}
\item[1]
        Let $\bb{X}$,$\bb{Y}$ and $\bb{Z}$ bet sets.
        Prove the transitivity of inclusion, that is,
        $$(\bb{X} \subseteq \bb{Y}) \land (\bb{Y} \subseteq \bb{Z}) \Ra \bb{X} \subseteq \bb{Z}$$
\begin{proof}
  Suppose exists an arbitary element $x \in \bb{X}$ then $x \in \bb{Y}$ by the definition of $\subseteq$,
  Hence, $x \in \bb{Z}$ cause $\bb{Y} \subseteq \bb{Z}$.
\end{proof}

\item[2]
        Verify the claims of Proposition 2.4
\begin{proof}
Suppose that $\bb{X}$, $\bb{Y}$, $\bb{Z}$ are subset of $\bb{U}$.
\begin{enumerate}
  \item[(i)]
        $\bb{X} \cup \bb{Y} := \{ x \in \bb{U} : (x \in \bb{X}) \lor (x \in \bb{Y}) \}$ and
        $\bb{Y} \cup \bb{Y} := \{ x \in \bb{U} : (x \in \bb{Y}) \lor (x \in \bb{X}) \}$.
        Clearly, those two expression are equal by their definition. It's similar to $\bb{X} \cap \bb{Y}$.
  \item[(ii)]
        $\cap$ and $\cup$ are associativity due to associativity of $\land$ and $\lor$
  \item[(iii)]
        First to prove
        $$\bb{X} \cup (\bb{Y} \cap \bb{Z}) \Ra
        (\bb{X} \cup \bb{Y}) \cap (\bb{X} \cup \bb{Z})$$

        By definition,

        $$\bb{X} \cup (\bb{Y} \cap \bb{Z}) \Ra
        \forall x \in \bb{U}: (x \in \bb{X}) \lor (x \in \bb{Y} \land x \in \bb{Z})
        \Ra \forall x \in \bb{U}: (x \in \bb{X} \lor x \in \bb{Y}) \land (x \in \bb{X} \lor x \in \bb{Z})$$

        prove of another direction is similarly.
  \item[(iv)]
        Suppose that $\bb{X} \neq \bb{Y}$ because the case $\bb{X} = \bb{Y}$ is trivial. Under this assumption have $\bb{X} \subset \bb{Y}$. So the proposition can be rewrited as

        $$\bb{X} \subset \bb{Y} \iff \bb{X} \cup \bb{Y} = \bb{Y}
           \iff \bb{X} \cap \bb{Y} = \bb{X}$$

        which is clearly definite true.
\end{enumerate}
\end{proof}

\item[3] Provide a complete proof of Proposition 2.7.

\begin{enumerate}
  \item[(i)]
        $(\bigcap_{\alpha} \bf{A}_{\alpha}) \cap
        (\bigcap_{\beta} \bf{B}_{\beta}) =
        \bigcap_{(\alpha, \beta)} \bf{A}_{\alpha} \cap \bf{B}_{\beta}$.

        $(\bigcup_{\alpha} \bf{A}_{\alpha}) \cup (\bigcup_{\beta} \bf{B}_{\beta}) =
         \bigcup_{(\alpha,\beta)} \bf{A}_{\alpha} \cup \bf{B}_{\beta}$ (associativity)


        \begin{proof}
          by definition we have
          \begin{align*}
            (\bigcap_{\alpha} \bf{A}_{\alpha}) \cap (\bigcap_{\beta} \bf{B}_{\beta})
            & = \{ x \in \bb{X}: \forall \alpha \in \sf{A}: x \in \bf{A}_{\alpha} \} \cap
              \{ x \in \bb{X}: \forall \beta \in \sf{B}: x \in \bf{B}_{\beta} \} \\
            & = \{ x \in \bb{X}: \forall \alpha \in \sf{A}, \forall \beta \in \sf{B}:
                           x \in \bf{A}_{\alpha} \land x \in \bf{B}_{\beta}\} \\
            & = \bigcap_{(\alpha,\beta)} \bf{A}_{\alpha} \cap \bf{B}_{\beta}
          \end{align*}

          Prove of $\bigcup$ is similarly.
        \end{proof}

  \item[(ii)]
        $(\bigcap_{\alpha} \bf{A}_{\alpha}) \cup (\bigcap_{\beta} \bf{B}_{\beta}) =
        \bigcap_{(\alpha,\beta)} \bf{A}_{\alpha} \cup \bf{B}_{\beta}$ \\
        $(\bigcup_{\alpha} \bf{A}_{\alpha}) \cap (\bigcup_{\beta} \bf{B}_{\beta}) =
        \bigcup_{(\alpha,\beta)} \bf{A}_{\alpha} \cap \bf{B}_{\beta}$ (distribuitivity)

        \begin{proof}
          By Proposition 2.4(iii) and the definition of $\bigcap$ and $\bigcup$ we have
          \begin{align*}
            (\sf{A}_{\alpha_{0}} \cap \sf{A}_{\alpha_{1}} \text{...}) \cup
            (\sf{B}_{\beta_{0}} \cap \sf{B}_{\beta_{1}} \text{...})
            &= ((\sf{A}_{\alpha_{0}} \cap \sf{A}_{\alpha_{1}} \text{...}) \cup \sf{B}_{\beta_{0}} \cap
              (\sf{A}_{\alpha_{0}} \cap \sf{A}_{\alpha_{1}} \text{...}) \cup \sf{B}_{\beta_{1}} \cap \text{...})
          \end{align*}

          Then $\bigcap_{(\alpha,\beta)} \sf{A}_{\alpha} \cup \sf{B}_{\beta}$ is got by apply Proposition 2.4(iii) again to right side of this equation.
          This method can be apply to the second equation of this proposition.
        \end{proof}
  \item[(iii)]
        $(\bigcap_{\alpha} \sf{A}_{\alpha})^{c} = \bigcup_{\alpha} \sf{A}^{c}_{\alpha}$ \\
        $(\bigcup_{\alpha} \sf{A}_{\alpha})^{c} = \bigcap_{\alpha} \sf{A}^{c}_{\alpha}$ (de Morgan's laws)

        \begin{proof}
          Write your proof here.
        \begin{align*}
        \end{align*}
        \end{proof}
\end{enumerate}

  \item[4] Let $\bf{X}$ and $\bf{Y}$ be nonempty sets. Show that
        $$\bf{X} \times \bf{Y} = \bf{Y} \times \bf{X} \iff \bf{X} = \bf{Y}$$
        \begin{proof}

        \end{proof}
        
\end{enumerate}


\end{document}
